\begin{centering}
    \subsection{Динамика СВП} % 29 58
\end{centering}

В настоящей работе предпологается, что корпус судна является абсолютно жестким, следовательно, его динамику можно описать несколькими уравнениями для поступательных и вращательных движений по каждой координате. Однако, на данном этапе исследования нас интересует только вертикальное движение и тангаж, поэтому большую часть уравнений можно исключить без потери общности. Также, при анализе устойчивости СВП будет использоваться упрощенное представление судна - гибкие ограждения воздушной подушки представлены как недеформируемый стержень, жестко прикрепленный к корпусу судна, чтобы исключить сложные и трудоемкие вычисления динамики и демпфирования гибкого ограждения. Такая модель называется эквивалентным стержнем и вводится для более простого описания момента вращения судна. С учетом всего вышсказаного для описания динамики СВП можно выписать второй закон Ньютона и момент вращения судна:
\begin{equation}\label{eq:acv_dynamics}
    \begin{gathered}
        m\dfrac{d^2H}{dt^2}=\mathbf{F}_{AC} + \mathbf{F}_{Wave} = \mathbf{F}\\
        I_z\dfrac{d^2\varphi}{dt^2}=\mathbf{M}_{AC} + \mathbf{M}_{Wave} = \mathbf{M}
    \end{gathered}
\end{equation}
Индекс $AC$ в (\ref{eq:acv_dynamics}) означает, что сила или момен действуют на корпус судна со стороны воздушной подушки, а $Wave$ от контакта корпуса судна с волной. Слагаемые в разложении силы и момента вычисляются путем интегрирования уравнений по времени на каждом шаге моделирования по аналитическим закономерностям, описанным далее.
