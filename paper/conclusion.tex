\begin{centering}
    \section*{Заключение}
\end{centering}
\addcontentsline{toc}{section}{Заключение}
\setcounter{section}{1}

В ходе исследования была проведен анализ литературы, на основе которого была разработана упрощенная математическая модель динамики судна на воздушной подушке. Эта модель была дополнительно усовершенствована и проинтегрирована неявным методом Эйлера по времени для проведения моделирования движения. Путем экспериментов были подобраны оптимальные гиперпараметры математической модели, что позволило улучшить точность моделирования и достоверно отразить реальные физические взаимодействия.

Далее с использованием полученной модели были проведены серии экспериментов с целью увеличения демпфирующих сил. Результаты этих экспериментов показали, что изменение демпфирующих сил влияет на динамику судна и его поведение на воздушной подушке. Можно утвержать, что СВП имеют возможности автоматического демпфирования для снижения параметров вертикальной качки.

Полученная модель представляет собой ценный инструмент для дальнейших исследований в области судостроения и подводного транспорта на воздушной подушке.

Исследования по данной теме можно продолжить, например разработав правило, которое будет регулировать коэффициент демпфирования {\it на ходу} в зависимости от накопленных измерениях (как это делает ПИД регулятор). Также можно попробовать обучить модель машинного обучения для опреления наиболее эффективного коэффициента демпфирования в зависимости от параметров окружающей системы. 