\begin{centering}
    \subsection{Вычислительные эксперименты}
\end{centering}

В данном разделе дипломной работы представлены результаты экспериментальных исследований, проведенных с целью изучения влияния различных факторов на поведение судна в условиях волнения моря. 

\begin{figure}[!hb]
    \centering
    % This file was created with tikzplotlib v0.10.1.
\begin{tikzpicture}

\definecolor{darkgray176}{RGB}{176,176,176}
\definecolor{darkorange25512714}{RGB}{255,127,14}
\definecolor{lightgray204}{RGB}{204,204,204}
\definecolor{steelblue31119180}{RGB}{31,119,180}

\begin{groupplot}[group style={group size=1 by 3}]
\nextgroupplot[
scaled x ticks=manual:{}{\pgfmathparse{#1}},
tick align=outside,
tick pos=left,
x grid style={darkgray176},
xmajorgrids,
xmin=0, xmax=16,
xtick style={color=black},
xticklabels={},
y grid style={darkgray176},
ylabel={Вертикальная перегрузка},
ymajorgrids,
ymin=0, ymax=0.4,
ytick style={color=black},
ytick={0,0.05,0.1,0.15,0.2,0.25,0.3,0.35,0.4},
yticklabels={0.00,0.05,0.10,0.15,0.20,0.25,0.30,0.35, 0.40},
scaled y ticks = false,
width=14cm,
height=7cm
]
\addplot [semithick, steelblue31119180, mark=*, mark size=3, mark options={solid}]
table {%
3 0.05
5 0.175
7 0.18
10 0.215
12 0.275
15 0.375
};
\addplot [semithick, darkorange25512714, mark=triangle*, mark size=3, mark options={solid}]
table {%
3 0.01
5 0.11
7 0.115
10 0.105
12 0.105
15 0.1
};

\nextgroupplot[
legend cell align={left},
legend style={
  fill opacity=0.8,
  draw opacity=1,
  text opacity=1,
  at={(0.03,0.97)},
  anchor=north west,
  draw=lightgray204
},
scaled x ticks=manual:{}{\pgfmathparse{#1}},
tick align=outside,
tick pos=left,
x grid style={darkgray176},
xmajorgrids,
xmin=0, xmax=16,
xtick style={color=black},
xticklabels={},
y grid style={darkgray176},
ylabel={Размах вертикальной качки [м]},
ymajorgrids,
ymin=0, ymax=0.2,
ytick style={color=black},
yticklabels={0, 0.00, 0.05, 0.10, 0.15, 0.20},
scaled y ticks = false,
width=14cm,
height=7cm
]
\addplot [semithick, steelblue31119180, mark=*, mark size=3, mark options={solid}]
table {%
3 0
5 0.02
7 0.025
10 0.06
12 0.1
15 0.19
};
\addlegendentry{обычное демпфирование}
\addplot [semithick, darkorange25512714, mark=triangle*, mark size=3, mark options={solid}]
table {%
3 0
5 0.018
7 0.019
10 0.05
12 0.065
15 0.06
};
\addlegendentry{усиленное демпфирование}

\nextgroupplot[
tick align=outside,
tick pos=left,
x grid style={darkgray176},
xlabel={Длина волны [м]},
xmajorgrids,
xmin=0, xmax=16,
xtick style={color=black},
y grid style={darkgray176},
ylabel={Тангаж [\,°\,]},
ymajorgrids,
ymin=0, ymax=6,
ytick style={color=black},
yticklabels={0, 0, 2, 4, 6},
scaled y ticks = false,
width=14cm,
height=7cm
]
\addplot [semithick, steelblue31119180, mark=*, mark size=3, mark options={solid}]
table {%
3 0
5 0.8
7 1.4
10 3.5
12 5.6
15 5
};
\addplot [semithick, darkorange25512714, mark=triangle*, mark size=3, mark options={solid}]
table {%
3 0
5 0.6
7 0.6
10 1.2
12 1.4
15 1.4
};
\end{groupplot}

\end{tikzpicture}

    \caption{Влияние увеличения коэффициента демпфирования на динамику судна на скорости 25 км/ч}\label{fig:plots_25}
\end{figure}

В ходе экспериментов было проведено множество испытаний с длиной волны от 2 до 16 метров, что позволило нам получить обширный объем данных для последующего анализа.
\begin{figure}[!hb]
    \centering
    % This file was created with tikzplotlib v0.10.1.
\begin{tikzpicture}

\definecolor{darkgray176}{RGB}{176,176,176}
\definecolor{darkorange25512714}{RGB}{255,127,14}
\definecolor{lightgray204}{RGB}{204,204,204}
\definecolor{steelblue31119180}{RGB}{31,119,180}

\begin{groupplot}[group style={group size=1 by 3}]
\nextgroupplot[
scaled x ticks=manual:{}{\pgfmathparse{#1}},
tick align=outside,
tick pos=left,
x grid style={darkgray176},
xmajorgrids,
xmin=0, xmax=16,
xtick style={color=black},
xticklabels={},
y grid style={darkgray176},
ylabel={Вертикальная перегрузка},
ymajorgrids,
ymin=0, ymax=0.4,
ytick style={color=black},
ytick={0,0.05,0.1,0.15,0.2,0.25,0.3,0.35,0.4},
yticklabels={0.00,0.05,0.10,0.15,0.20,0.25,0.30,0.35, 0.40},
scaled y ticks = false,
width=14cm,
height=7cm
]
\addplot [semithick, steelblue31119180, mark=*, mark size=3, mark options={solid}]
table {%
3 0.04
5 0.275
7 0.26
10 0.355
12 0.35
15 0.36
};
\addplot [semithick, darkorange25512714, mark=triangle*, mark size=3, mark options={solid}]
table {%
3 0.02
5 0.175
7 0.13
10 0.13
12 0.15
15 0.15
};

\nextgroupplot[
legend cell align={left},
legend style={
  fill opacity=0.8,
  draw opacity=1,
  text opacity=1,
  at={(0.03,0.97)},
  anchor=north west,
  draw=lightgray204
},
scaled x ticks=manual:{}{\pgfmathparse{#1}},
tick align=outside,
tick pos=left,
x grid style={darkgray176},
xmajorgrids,
xmin=0, xmax=16,
xtick style={color=black},
xticklabels={},
y grid style={darkgray176},
ylabel={Размах вертикальной качки [м]},
ymajorgrids,
ymin=0, ymax=0.2,
ytick style={color=black},
yticklabels={0, 0.00, 0.05, 0.10, 0.15, 0.20},
scaled y ticks = false,
width=14cm,
height=7cm
]
\addplot [semithick, steelblue31119180, mark=*, mark size=3, mark options={solid}]
table {%
3 0
5 0.01
7 0.02
10 0.04
12 0.05
15 0.085
};
\addlegendentry{обычное демпфирование}
\addplot [semithick, darkorange25512714, mark=triangle*, mark size=3, mark options={solid}]
table {%
3 0
5 0.005
7 0.01
10 0.015
12 0.029
15 0.05
};
\addlegendentry{усиленное демпфирование}

\nextgroupplot[
tick align=outside,
tick pos=left,
x grid style={darkgray176},
xlabel={Длина волны [м]},
xmajorgrids,
xmin=0, xmax=16,
xtick style={color=black},
y grid style={darkgray176},
ylabel={Тангаж [\,°\,]},
ymajorgrids,
ymin=0, ymax=6,
ytick style={color=black},
yticklabels={0, 0, 2, 4, 6},
scaled y ticks = false,
width=14cm,
height=7cm
]
\addplot [semithick, steelblue31119180, mark=*, mark size=3, mark options={solid}]
table {%
3 0
5 0.4
7 0.5
10 1.5
12 2.5
15 4
};
\addplot [semithick, darkorange25512714, mark=triangle*, mark size=3, mark options={solid}]
table {%
3 0
5 0.35
7 0.5
10 0.9
12 1
15 1.2
};
\end{groupplot}

\end{tikzpicture}

    \caption{Влияние увеличения коэффициента демпфирования на динамику судна на скорости 50 км/ч}\label{fig:plots_50}
\end{figure}

Во время каждого эксперимента регистрировались три основных параметра: вертикальная качка, тангаж и вертикальный размах. Эти параметры являются ключевыми для понимания динамики судна в условиях волнения и позволяют оценить его устойчивость и управляемость.

Для более детального изучения влияния скорости судна на его поведение в условиях волнения было проведено две серии испытаний. В первой серии испытаний скорость судна была 25 км/ч, а во второй серии она увеличилась в два раза. Это позволило нам выявить зависимость между скоростью судна и его реакцией на волнение моря.

На графиках сравниваются две модели судна - первая реализует математическую модель, описанную в настоящей работе, вторая является аналогичной, но с увеличенным в 5 раз коэффициентом демпфирования (на практике без внедрения в конструктивные особенности судна такого добиться практически невозможно).

По результатам экспериментов можно сделать несколько важных выводов. Во-первых, видна зависимость демпфирующей силы от высоты зазора между гибким огражением судна и водой (Рис. \ref{fig:plots_50}, график перегрузки) - чем больше длина волны при неизменной амплитуде, тем меньше будет высота зазора при прохождении данной волны. В экспериментах получилось полностью избежать данного эффекта путем увеличения коэффициента демпфирования. Также можно заметить, что увеличение демпфирубщих сил поможет судну с исходными параметрами из Таблицы \ref{tab:total_params} улучшить большинство характеристик, связанных с устойчивостью, независимо как от длины волны, так и от скорости самого судна.

В целом, можно сказать что СВП имеет возможности автоматического демпфирования как минимум посредством корректирования коэффициента демпфирования в разных состояних волнения и скорости судна. Способы варьирования коэффицинта демпфирования не являются целью настоящей работы и являются задачей проектировщиков и CFD инженеров.