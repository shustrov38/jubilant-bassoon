\begin{centering}
    \section*{Введение}
\end{centering}
\addcontentsline{toc}{section}{Введение}
\setcounter{section}{1}

\begin{centering}
    \subsubsection*{Обзор по теме исследования и актуальность}
\end{centering}

Суда, работающие на воздушной подушке, поддерживаются над поверхностью (водой, грунтом или льдом) за счет избыточного давления воздуха, который непрерывно поступает под днище воздушной подушки. Эти новаторские транспортные средства обладают рядом преимуществ перед традиционными кораблями и лодками, что делает их все более популярными в наше время. В частности, суда на воздушной подушке способны развивать гораздо большую скорость, чем обычные корабли, что позволяет значительно сократить время путешествия и повысить эффективность транспортировки грузов и пассажиров. Кроме того, они обладают отличной маневренностью и способностью преодолевать препятствия, такие как льды или водоросли, что делает их идеальным выбором для использования в условиях суровых климатических условий.

Однако, главной проблемой СВП на данный момент являются вертикальные колебания во время движения\cite{fein1975dynamic}. Текущее состояние исследований вертикальных колебаний СВП зависит от класса судна. Для СВП с классическим юбобным ограждением получен аналитический критерий устойчивости \cite{mantle1975technical}, согласно которому колебания можно уменьшить увеличивая расход вентиряторов в подушку, одновременно с этим уменьшая длину судна и давление в ВП. Однако данное условие не учитывает собственную динамику гибкого огражения, которое может быть обсчитано с помощью серии самостоятельных CFD рассчетов. Судно другого типа было изучено в работе\cite{chung2000linear}. В \cite{демешко1992проектирование} показано, что для уменьшенения низкочастотных колебаний СВП необходимо увеличение давления в гибком ограждении ВП, что ведет к увеличению расхода воздуха, но внеднение такого метода решения сильно снижает мореходные качества судна.

Скеговые СВП используют жесткие скеги в качестве бокового ограждения зоны воздушной подушки, поэтому автоколебания не могут возникнуть для таких судов. Тем не менее СВПС могут испытывать колебания, известные как {\it движение по булыжной мостовой}. Данное явление представляет собой вертикальные колебания, вызванные резонансными эффектами в воздушной подушке. Они обычно возникают при небольшом волнении\cite{faltinsen1997effect}.

Значительная часть современных работ, посвященных исследованию динамики СВП использует численные методы для точного интегрирования систем дифференциальных уравнений. Обсчет взаимодействия ВП и гибкого ограждения методом RANS был проведен в работе \cite{cole2018numerical}. Также применялись методы ALE\cite{challa2013finite} и SPH\cite{ozbulut2014numerical}.

\begin{centering}
    \subsubsection*{Постановка задачи}
\end{centering}

Целью настоящей работы является проработка возможности снижения вертикальных колебаний судна на воздушной подушке посредством изменения демпфирующих свойств зоны повышенного давления. В соответствии с поставленной целью выбран ряд задач:
\vspace*{5mm}
\begin{enumerate}
    \item Построение математической модели судна на воздушной подушке скегового типа с учетом демпфирования, обусловленного вертикальны движением судна
    \item Апробация модели на примере конкретного судна
    \item Проведение вычислительных экспериментов с целью выявления влияния коэффицента демпфирования ВП на характеристики вертикальной качки судна
\end{enumerate}