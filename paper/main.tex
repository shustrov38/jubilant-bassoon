\documentclass{article}%

\usepackage[14pt]{extsizes}%
\usepackage[left=3.5cm,right=2.5cm,top=2.5cm,bottom=2.5cm,footskip=10mm]{geometry}%

\usepackage{cmap}%
\usepackage[T1,T2A]{fontenc}%
\usepackage[utf8]{inputenc}%
\usepackage{csquotes}%
\usepackage[russian]{babel}%

\usepackage{amsmath, amsfonts, amssymb, mathtools, systeme}%
\usepackage{graphicx, float}%
\usepackage{caption}%
\usepackage{setspace}%
\usepackage{array}%
\usepackage{booktabs}%

\usepackage{tikz}%
\usepackage{pgfplots}%

% вычисление значения в точке
\newcommand\at[2]{\left.#1\right|_{#2}}

\numberwithin{equation}{subsection}

\let\oldsection\section% Store \section
\renewcommand{\section}{% Update \section
  \renewcommand{\theequation}{\thesection.\arabic{equation}}% Update equation number
  \oldsection}% Regular \section
\let\oldsubsection\subsection% Store \subsection
\renewcommand{\subsection}{% Update \subsection
  \renewcommand{\theequation}{\thesubsection.\arabic{equation}}% Update equation number
  \oldsubsection}% Regular \subsection

\begin{document}
\section{Динамика СВП}
Исходная система уравнений, описывающая динамику СВП:
\begin{equation}
\begin{gathered}
    m\dfrac{d^2H}{dt^2}=\mathbf{F}\\
    \dfrac{dp}{dt}=\dfrac{np_a}{W}\left(Q_{in} - Q_{out} -\dfrac{dW}{dt}\right)\\
    I_z\dfrac{d^2\varphi}{dt^2}=\mathbf{M}
\end{gathered}
\label{eq:1.1}
\end{equation}

\section{Расходно напорная характеристика}
\begin{table}[h]
    \caption{Исходные параметры центробежного нагнетателя, округленные до целой части. Точные параметры можно найти в {\it initials.xlsx}}
    \centering
    \begin{tabular}{@{}l*{10}{l}@{}}
        \toprule
        $p$, Па & 2809 & 2965 & 2902 & 2715 & 2497 & 2325 & 2060 & 1280 & 593 \\
        \midrule
        $Q, \frac{\text{м}^3}{\text{c}}$ & 0 & 6 & 11 & 17 & 22 & 25 & 28 & 34 & 38 \\
        \bottomrule
    \end{tabular}
\end{table}
\begin{figure}[H]
    \centering
    % This file was created with tikzplotlib v0.10.1.
\begin{tikzpicture}

\definecolor{darkgray176}{RGB}{176,176,176}
\definecolor{lightgray204}{RGB}{204,204,204}
\definecolor{steelblue31119180}{RGB}{31,119,180}

\begin{axis}[
legend cell align={left},
legend style={fill opacity=0.8, draw opacity=1, text opacity=1, draw=lightgray204},
tick align=outside,
tick pos=left,
x grid style={darkgray176},
xlabel={Объемный расход воздуха в ВП [$Q$, $\dfrac{\text{м}^3}{\text{с}}$]},
xlabel near ticks,
xmin=0, xmax=47.9664362355135,
xtick style={color=black},
y grid style={darkgray176},
ylabel={Избыточное давление ВП [$p$, Па]},
ylabel near ticks,
ymin=473.424819454775, ymax=3103.40709363445,
ytick style={color=black},
width=14cm,
height=8cm
]
\addplot [semithick, black, opacity=0.7]
table {%
0 2808.80274448953
5.58329944639905 2964.84734140562
11.1665988927981 2902.42950263918
16.7498983391972 2715.17598633988
22.3331977855962 2496.71355065736
25.1248475087957 2325.06449404967
27.9164972319952 2059.78867929232
33.4997966783943 1279.5656947119
37.9664362355135 592.969468281124
};
\addlegendentry{Табличные данные}
\addplot [semithick, steelblue31119180]
table {%
0 2770.84648143807
0.383499355914278 2789.02622646661
0.766998711828556 2806.39522773118
1.15049806774283 2822.95348523176
1.53399742365711 2838.70099896837
1.91749677957139 2853.637768941
2.30099613548567 2867.76379514965
2.68449549139994 2881.07907759432
3.06799484731422 2893.58361627502
3.4514942032285 2905.27741119174
3.83499355914278 2916.16046234448
4.21849291505706 2926.23276973324
4.60199227097133 2935.49433335802
4.98549162688561 2943.94515321883
5.36899098279989 2951.58522931566
5.75249033871417 2958.41456164851
6.13598969462844 2964.43315021738
6.51948905054272 2969.64099502227
6.902988406457 2974.03809606319
7.28648776237128 2977.62445334013
7.66998711828555 2980.40006685309
8.05348647419983 2982.36493660207
8.43698583011411 2983.51906258708
8.82048518602839 2983.8624448081
9.20398454194267 2983.39508326515
9.58748389785694 2982.11697795822
9.97098325377122 2980.02812888732
10.3544826096855 2977.12853605243
10.7379819655998 2973.41819945357
11.1214813215141 2968.89711909073
11.5049806774283 2963.56529496391
11.8884800333426 2957.42272707311
12.2719793892569 2950.46941541834
12.6554787451712 2942.70535999959
13.0389781010854 2934.13056081686
13.4224774569997 2924.74501787015
13.805976812914 2914.54873115947
14.1894761688283 2903.5417006848
14.5729755247426 2891.72392644616
14.9564748806568 2879.09540844354
15.3399742365711 2865.65614667694
15.7234735924854 2851.40614114637
16.1069729483997 2836.34539185182
16.4904723043139 2820.47389879328
16.8739716602282 2803.79166197078
17.2574710161425 2786.29868138429
17.6409703720568 2767.99495703382
18.0244697279711 2748.88048891938
18.4079690838853 2728.95527704096
18.7914684397996 2708.21932139856
19.1749677957139 2686.67262199219
19.5584671516282 2664.31517882183
19.9419665075424 2641.1469918875
20.3254658634567 2617.16806118919
20.708965219371 2592.3783867269
21.0924645752853 2566.77796850064
21.4759639311996 2540.36680651039
21.8594632871138 2513.14490075617
22.2429626430281 2485.11225123797
22.6264619989424 2456.26885795579
23.0099613548567 2426.61472090964
23.3934607107709 2396.14984009951
23.7769600666852 2364.87421552539
24.1604594225995 2332.78784718731
24.5439587785138 2299.89073508524
24.9274581344281 2266.18287921919
25.3109574903423 2231.66427958917
25.6944568462566 2196.33493619517
26.0779562021709 2160.19484903719
26.4614555580852 2123.24401811523
26.8449549139994 2085.4824434293
27.2284542699137 2046.91012497939
27.611953625828 2007.5270627655
27.9954529817423 1967.33325678763
28.3789523376566 1926.32870704578
28.7624516935708 1884.51341353996
29.1459510494851 1841.88737627016
29.5294504053994 1798.45059523638
29.9129497613137 1754.20307043862
30.2964491172279 1709.14480187689
30.6799484731422 1663.27578955117
31.0634478290565 1616.59603346148
31.4469471849708 1569.10553360781
31.8304465408851 1520.80428999017
32.2139458967993 1471.69230260854
32.5974452527136 1421.76957146294
32.9809446086279 1371.03609655336
33.3644439645422 1319.4918778798
33.7479433204564 1267.13691544226
34.1314426763707 1213.97120924075
34.514942032285 1159.99475927525
34.8984413881993 1105.20756554579
35.2819407441136 1049.60962805234
35.6654401000278 993.200946794911
36.0489394559421 935.981521773506
36.4324388118564 877.951352988125
36.8159381677707 819.110440438765
37.1994375236849 759.458784125427
37.5829368795992 698.996384048112
37.9664362355135 637.723240206818
};
\addlegendentry{$f(q)=Aq^2+Bq+C$}
\end{axis}

\end{tikzpicture}

    \caption{Расходно напорная характеристика $p(Q_{in})$. Аппроксимация МНК квадратичной функцией $f(q)=-2.756q^2 + 48.46q + 2771$.}
\end{figure}
Для выполнения условия устойчивости судна в предположении отсутсвия волнения, необходимо:
\begin{equation}
    \dfrac{1}{2}Q_0 - \at{\dfrac{\partial Q}{\partial p}}{0}p_0 > 0, \text{ где } (Q_0, p_0) \text{ точка на нисходящей ветви РНХ}
\end{equation} 
Преобразуем $p(Q_{in})$ для получения функции $Q_{in}(p)$, решив уравнение $f(q)=p$ и выбрав решения на нисходящей ветке. Получим:
\begin{equation}
    Q_{in}(p)=\dfrac{-B-\sqrt{B^2-4A(C-p)}}{2A}, \text{ где } A, B, C \text{ из } f(q)
\end{equation}
Объмный расход воздуха $Q_{out}$, вытекающего из зоны ВП, может быть вычислено слующим образом:
\begin{equation}
    Q_{out}=Q_{out}(p, S_{gap})=\chi\sqrt{\dfrac{2p}{\rho_{a}}}S_{gap}
\end{equation}
Объмный расход воздуха $Q_{in}$, нагнетаемого вентиляторами в зону ВП, вычисляется с помощью РНХ вентилятора:
\begin{equation}
    Q_{in}=Q_{in}(p)
\end{equation}
\begin{figure}[H]
    \centering
    % This file was created with tikzplotlib v0.10.1.
\begin{tikzpicture}

\definecolor{darkgray176}{RGB}{176,176,176}
\definecolor{darkorange25512714}{RGB}{255,127,14}
\definecolor{lightgray204}{RGB}{204,204,204}
\definecolor{steelblue31119180}{RGB}{31,119,180}

\begin{axis}[
legend cell align={left},
legend style={fill opacity=0.8, draw opacity=1, text opacity=1, draw=lightgray204},
tick align=outside,
tick pos=left,
x grid style={darkgray176},
xlabel={Избыточное давление ВП [$p$, Па]},
xlabel near ticks,
xmin=0, xmax=3083.86481468021,
xtick style={color=black},
xtick distance=500,
y grid style={darkgray176},
ylabel={Объемный расход воздуха ВП [$Q$, $\dfrac{\text{м}^3}{\text{с}}$]},
ylabel near ticks,
ymin=0, ymax=43,
ytick style={color=black},
width=14cm,
height=9cm
]
\addplot [semithick, steelblue31119180, opacity=0.1, forget plot]
table {%
0 -24.1132441653123
30.1410587341435 -23.9466389327784
60.2821174682871 -23.7791814998747
90.4231762024307 -23.610858654021
120.564234936574 -23.4416568376514
150.705293670718 -23.2715621354711
180.846352404861 -23.1005602611001
210.987411139005 -22.9286365430695
241.128469873148 -22.755775910131
271.269528607292 -22.5819628758374
301.410587341435 -22.4071815223505
331.551646075579 -22.2314154834275
361.692704809723 -22.0546479265351
391.833763543866 -21.8768615340358
421.97482227801 -21.6980384833864
452.115881012153 -21.5181604262856
482.256939746297 -21.3372084667007
512.39799848044 -21.1551631376994
542.539057214584 -20.9720043770059
572.680115948727 -20.7877115011956
602.821174682871 -20.6022631784326
632.962233417015 -20.4156373996492
663.103292151158 -20.2278114480584
693.244350885302 -20.0387618668772
723.385409619445 -19.8484644251332
753.526468353589 -19.6568940814122
783.667527087732 -19.464024945394
813.808585821876 -19.2698302370079
843.949644556019 -19.0742822430275
874.090703290163 -18.8773522709027
904.231762024306 -18.6790105996151
934.37282075845 -18.4792264273138
964.513879492594 -18.2779678154748
994.654938226737 -18.0752016292943
1024.79599696088 -17.8708934740022
1054.93705569502 -17.6650076267488
1085.07811442917 -17.4575069636835
1115.21917316331 -17.2483528818023
1145.36023189745 -17.0375052151002
1175.5012906316 -16.8249221445112
1205.64234936574 -16.6105601010644
1235.78340809989 -16.3943736616215
1265.92446683403 -16.1763154364885
1296.06552556817 -15.9563359481135
1326.20658430232 -15.7343834999925
1356.34764303646 -15.5104040347975
1386.4887017706 -15.284340980625
1416.62976050475 -15.0561350841259
1446.77081923889 -14.8257242291214
1476.91187797303 -14.5930432391322
1507.05293670718 -14.3580236620424
1537.19399544132 -14.1205935348813
1567.33505417546 -13.8806771264321
1597.47611290961 -13.6381946550591
1627.61717164375 -13.3930619787726
1657.7582303779 -13.1451902541181
1687.89928911204 -12.8944855599693
1718.04034784618 -12.6408484817088
1748.18140658033 -12.3841736505739
1778.32246531447 -12.1243492321162
1808.46352404861 -11.8612563567292
1838.60458278276 -11.5947684840189
1868.7456415169 -11.3247506913731
1898.88670025104 -11.0510588753788
1929.02775898519 -10.7735388526746
1959.16881771933 -10.4920253443094
1989.30987645347 -10.2063408246146
2019.45093518762 -9.91629421181934
2049.59199392176 -9.62167937297714
2079.7330526559 -9.32227340996624
2109.87411139005 -9.01783468605888
2140.01517012419 -8.70810054338757
2170.15622885834 -8.39278464999017
2200.29728759248 -8.0715739001989
2230.43834632662 -7.7441247728724
2260.57940506077 -7.41005902685606
2290.72046379491 -7.06895858000031
2320.86152252905 -6.72035937409369
2351.0025812632 -6.36374396890471
2381.14363999734 -5.99853252793608
2411.28469873148 -5.62407174723212
2441.42575746563 -5.23962112270392
2471.56681619977 -4.84433572951856
2501.70787493391 -4.43724436551243
2531.84893366806 -4.01722143534673
2561.9899924022 -3.58295023430422
2592.13105113635 -3.13287417955702
2622.27210987049 -2.66513076881048
2652.41316860463 -2.17746014324924
2682.55422733878 -1.66707518949049
2712.69528607292 -1.13047134016334
2742.83634480706 -0.563137861281627
2772.97740354121 0.0408999959690005
2803.11846227535 0.689847591568909
2833.25952100949 1.39551670965454
2863.40057974364 2.17603328049188
2893.54163847778 3.06190325477623
2923.68269721192 4.11258637818413
2953.82375594607 5.48146509862148
2983.96481468021 8.64574323982497
};
\addplot [semithick, steelblue31119180]
table {%
0 41.6966984468798
30.1400486331334 41.5300988119033
60.2800972662669 41.3626470341138
90.4201458994004 41.1943299022763
120.560194532534 41.0251338602176
150.700243165667 40.8550449940846
180.840291798801 40.6840490189907
210.980340431934 40.5121312650146
241.120389065068 40.3392766625117
271.260437698201 40.1654697266985
301.400486331334 39.9906945414622
331.540534964468 39.8149347423513
361.680583597601 39.6381734986921
391.820632230735 39.4603934947789
421.960680863868 39.2815769100764
452.100729497002 39.1017053983707
482.240778130135 38.9207600658007
512.380826763269 38.738721447694
542.520875396402 38.5555694841283
572.660924029536 38.3712834941308
602.800972662669 38.1858421484208
632.941021295802 37.9992234405961
663.081069928936 37.8114046566506
693.221118562069 37.6223623427052
723.361167195203 37.4320722708207
753.501215828336 37.2405094027539
783.64126446147 37.0476478515011
813.781313094603 36.8534608404629
843.921361727737 36.6579206600477
874.06141036087 36.4609986215153
904.201458994003 36.2626650078422
934.341507627137 36.0628890213701
964.48155626027 35.8616387279778
994.621604893404 35.6588809974889
1024.76165352654 35.4545814399998
1054.90170215967 35.2487043377838
1085.0417507928 35.0412125723868
1115.18179942594 34.8320675464947
1145.32184805907 34.6212291001066
1175.4618966922 34.4086554204979
1205.60194532534 34.1943029454021
1235.74199395847 33.9781262587748
1265.8820425916 33.7600779784362
1296.02209122474 33.5401086348019
1326.16213985787 33.3181665398245
1356.30218849101 33.0941976451621
1386.44223712414 32.8681453884704
1416.58228575727 32.6399505265812
1446.72233439041 32.4095509541721
1476.86238302354 32.1768815063554
1507.00243165667 31.9418737434076
1537.14248028981 31.7044557156247
1567.28252892294 31.4645517060137
1597.42257755607 31.2220819482105
1627.56262618921 30.9769623166482
1657.70267482234 30.7291039855611
1687.84272345547 30.4784130529081
1717.98277208861 30.2247901246993
1748.12282072174 29.9681298545085
1778.26286935487 29.7083204321209
1808.40291798801 29.4452430142753
1838.54296662114 29.1787710892803
1868.68301525427 28.9087697658638
1898.82306388741 28.6350949749153
1928.96311252054 28.3575925707106
1959.10316115367 28.0760973157043
1989.24320978681 27.7904317299057
2019.38325841994 27.5004047820839
2049.52330705307 27.2058103953866
2079.66335568621 26.9064257341582
2109.80340431934 26.6020092314796
2139.94345295248 26.292298307792
2170.08350158561 25.9770067193324
2200.22355021874 25.6558214602022
2230.36359885188 25.3283991226427
2260.50364748501 24.9943615950001
2290.64369611814 24.6532909438367
2320.78374475128 24.3047232827184
2350.92379338441 23.9481413711035
2381.06384201754 23.5829656062537
2411.20389065068 23.2085429599498
2441.34393928381 22.8241332560983
2471.48398791694 22.4288919636667
2501.62403655008 22.0218483582116
2531.76408518321 21.6018774306591
2561.90413381634 21.1676632052014
2592.04418244948 20.71765001878
2622.18423108261 20.2499765495404
2652.32427971574 19.7623844836521
2682.46432834888 19.2520887767881
2712.60437698201 18.7155877089977
2742.74442561514 18.1483745958667
2772.88447424828 17.5444806858157
2803.02452288141 16.8957099666188
2833.16457151454 16.1902662305493
2863.30462014768 15.4100519805108
2893.44466878081 14.5246205381449
2923.58471741394 13.4746652918666
2953.72476604708 12.1074099887393
2983.86481468021 9.03171832725908
};
\addlegendentry{$Q_{in}(p)$}
\addplot [semithick, darkorange25512714]
table {%
0 0
30.1400486331334 2.89471260031655
60.2800972662669 4.09374181853996
90.4201458994004 5.01378929705809
120.560194532534 5.78942520063311
150.700243165667 6.47277414963299
180.840291798801 7.09056882278062
210.980340431934 7.65868965744271
241.120389065068 8.18748363707992
271.260437698201 8.68413780094966
301.400486331334 9.15388498858896
331.540534964468 9.60067557116403
361.680583597601 10.0275785941162
391.820632230735 10.437034708173
421.960680863868 10.831022783562
452.100729497002 11.2111736930828
482.240778130135 11.5788504012662
512.380826763269 11.9352058069115
542.520875396402 12.2812254556199
572.660924029536 12.6177596953737
602.800972662669 12.945548299266
632.941021295802 13.2652396060931
663.081069928936 13.5774056006842
693.221118562069 13.8825539395287
723.361167195203 14.1811376455612
753.501215828336 14.4735630015828
783.64126446147 14.760196035257
813.781313094603 15.0413678911743
843.921361727737 15.3173793148854
874.06141036087 15.5885044219935
904.201458994003 15.8549938868781
934.341507627137 16.1170776566775
964.48155626027 16.3749672741598
994.621604893404 16.6288578762414
1024.76165352654 16.8789299218484
1054.90170215967 17.1253506926165
1085.0417507928 17.3682756018993
1115.18179942594 17.6078493411877
1145.32184805907 17.8442068879621
1175.4618966922 18.0774743949155
1205.60194532534 18.3077699771779
1235.74199395847 18.5352044114892
1265.8820425916 18.7598817590655
1296.02209122474 18.9818999220982
1326.16213985787 19.2013511423281
1356.30218849101 19.418322448899
1386.44223712414 19.6328960616575
1416.58228575727 19.8451497551991
1446.72233439041 20.0551571882324
1476.86238302354 20.2629882022159
1507.00243165667 20.4687090926998
1537.14248028981 20.6723828563618
1567.28252892294 20.8740694163461
1597.42257755607 21.0738258281895
1627.56262618921 21.2717064683419
1657.70267482234 21.4677632070444
1687.84272345547 21.6620455671241
1717.98277208861 21.8546008700821
1748.12282072174 22.0454743706962
1778.26286935487 22.2347093812232
1808.40291798801 22.4223473861656
1838.54296662114 22.6084281484653
1868.68301525427 22.7929898078938
1898.82306388741 22.9760689723281
1928.96311252054 23.1577008025324
1959.10316115367 23.3379190909996
1989.24320978681 23.5167563353553
2019.38325841994 23.6942438067752
2049.52330705307 23.870411613823
2079.66335568621 24.0452887620791
2109.80340431934 24.2189032098937
2139.94345295248 24.3912819205681
2170.08350158561 24.5624509112398
2200.22355021874 24.7324352987225
2230.36359885188 24.9012593425298
2260.50364748501 25.0689464852905
2290.64369611814 25.2355193907474
2320.78374475128 25.400999979513
2350.92379338441 25.5654094627419
2381.06384201754 25.7287683738667
2411.20389065068 25.891096598532
2441.34393928381 26.052413402849
2471.48398791694 26.2127374600856
2501.62403655008 26.372086875896
2531.76408518321 26.5304792121861
2561.90413381634 26.6879315097044
2592.04418244948 26.8444603094401
2622.18423108261 27.0000816729049
2652.32427971574 27.1548112013684
2682.46432834888 27.3086640541129
2712.60437698201 27.4616549657669
2742.74442561514 27.6137982627754
2772.88447424828 27.7651078790573
2803.02452288141 27.9155973708986
2833.16457151454 28.0652799311276
2863.30462014768 28.2141684026126
2893.44466878081 28.3622752911225
2923.58471741394 28.5096127775852
2953.72476604708 28.6561927297797
2983.86481468021 28.8020267134921
};
\addlegendentry{$Q_{out}(p, S_{gap})$}
\end{axis}

\end{tikzpicture}

    \caption{Расходно напорная характеристика $Q_{in}(p)$ и истечение $Q_{out}(p, S_{gap})$ при высоте зазора в 1 см.}
\end{figure}

\section{Удары волн о корпус}
Для учета удар волн о корпус будем использовать {\it метод плоских поперечных сечений}. Для этого необходимо разбить корпус на $N$ сечений (чем больше $N$, тем точнее будут вычисления). Тогда расстояние между сечениями:
\begin{equation}
    d_L = \dfrac{L}{N}
\end{equation}
Для каждого сечения необходимо вычислить положение свободной поверхности $y$. На данном этапе $y=0$. Если в каком то из сечений вода касается корпуса судна ($y>0$), то пятно контакта имеет площадь:
\begin{equation}
    S_{wash} = S_{wash}(d_L) = d_L \cdot b
\end{equation}
Тогда сила, действующая на пятно контакта, определяется как:
\begin{equation}
    F_{contact} = \theta\cdot\rho_{w} \dfrac{V^2}{2}S_{wash}, \text{ где } V = \sqrt{V_x^2 + V_y^2}
\end{equation}
Таким образом получим выражение для полной силы, которую оказывает удар волны:
\begin{equation}
    F_{wave} = \int\limits_{\mathcal{R}_{+}^{1}}F_{contact}\,\mathrm{d}d_L
\end{equation}
Также необходимо вычислить момент вращения судна под влиянием удара волны о корпус:
\begin{equation}
    M_{contact} = \int\limits_{0}^{L}\int\limits_{\mathcal{R}_{+}^{1}} d_{segment} \cdot F_{contact}\,\mathrm{d}d_L\mathrm{d}x
\end{equation}
где $d_{segment}$ - знаковое расстояние от центра тяжести до конкретного сечения.
\begin{equation}
    d_{segment}(x) = x - \dfrac{L}{2}
\end{equation}
если $d_{segment} > 0$, то сечение находится в носовой части судна относительно центра тяжести, а при $d_{segment} < 0$ в кормовой.

\subsection{Влияние на объем ВП и площадь зазора}
Объем ВП под действием волны $y(x)$ вычисляется сделующим образом:
\begin{equation}
    W_{shaped} = \int\limits_{x_c + 0}^{x_c + L}\int\limits_{\mathcal{R}_{+}^{1}}S_{wash}\left[H - y(x)\right]\,\mathrm{d}d_L\mathrm{d}x
\end{equation}
где $x_c$ - координата центра тяжести судна по оси $OX$, а $H-y(x)$ высота свободного пространства в сечении. Для облегчения задания площади зазора введем следующий индикатор:
\begin{equation}
    I(x) = 
    \begin{cases*}
        0.01, &H-d > y(x)\\
        0,&\text{иначе}
    \end{cases*}
\end{equation}
Индикатор $I(x)$ равен зазору в 1 см, когда нижняя часть ВП (центр тяжести $H$ минус клиренс судна $d$) находится выше волны $y(x)$. Тогда площадь зазора определяется следующим образом:
\begin{equation}
    S_{gap} = \int\limits_{x_c + 0}^{x_c + L}\int\limits_{\mathcal{R}_{+}^{1}}d_L \cdot I(x)\,\mathrm{d}d_L\mathrm{d}x
\end{equation}

\section{Симуляция}
Распишем $\mathbf{F}$ и $\mathbf{M}$ из \ref{eq:1.1} с учетом знаков:
\begin{equation}
\begin{gathered}
    F = pS - mg + F_{wave}\\
    M = pS\cdot l_{AC} + M_{contact}
\end{gathered}
\end{equation}
Также запишем уравнение, описывающее изменение объема ВП с учетом того, что волна может занимать часть объема ВП:
\begin{equation}
    \dfrac{dW}{dt}=S\dfrac{dH}{dt}+S\cdot l_{AC}\dfrac{d\varphi}{dt}-\dfrac{d(W-W_{shaped})}{dt}\\
    \label{eq:4.2}
\end{equation}
Последнее слагаемое в \ref{eq:4.2} описывает скорость изменения объема волны в ВП ($W_{wave} = W - W_{shaped}$).
\newline
Преобразуем \ref{eq:1.1}, заменив дифференциальные уравнения второго порядка уравнениями первого порядка. Также, допишем уравнение, описывающее изменение объема ВП:
\begin{equation}
\begin{gathered}
    \dfrac{dV_y}{dt}=\dfrac{pS-mg+F_{wave}}{m}\\
    \dfrac{dH}{dt} = V_y\\
    \dfrac{dW}{dt}=S\dfrac{dH}{dt}+S\cdot l_{AC}\dfrac{d\varphi}{dt}-\dfrac{d(W-W_{shaped})}{dt}\\
    \dfrac{dp}{dt}=\dfrac{np_a}{W}\left(Q_{in} - Q_{out} -\dfrac{dW}{dt}\right)\\
    \dfrac{dV_{\varphi}}{dt}=\dfrac{pS\cdot l_{AC} + M_{contact}}{I_z}\\
    \dfrac{d\varphi}{dt} = V_{\varphi}
\end{gathered}
\end{equation}

\subsection{Шаг симуляции}
В качестве метода численного интегрирования выбран метод {\it Рунге-Кутты четвертого порядка} в следующей формулировке:
\newline\par
Рассмотрим задачу Коши для системы обыкновенных дифференциальных уравнений первого порядка (далееий первого порядка (далее $\mathbf{y}, \mathbf{f}, \mathbf{k}_i \in \mathcal{R}^n, x, h \in \mathcal{R}^1$)
\begin{equation}
    \mathbf{y}'=\mathbf{f}(x, \mathbf{y}), \quad \mathbf{y}(x_0)=\mathbf{y}_0
\end{equation}
Тогда приближенное значение в последующих точдках вычисляется по итерационной формуле:
\begin{equation}
    \mathbf{y}_{n+1}=\mathbf{y}_n + \dfrac{h}{6}(\mathbf{k}_1 + 2\mathbf{k}_2 + 2\mathbf{k}_3 + \mathbf{k}_4)
\end{equation}
Вычисление нового значения происходит в четыре стадии:
\begin{equation}
\begin{gathered}
    \mathbf{k}_1 = \mathbf{f}(x_n, \mathbf{y}_n)\\
    \mathbf{k}_2 = \mathbf{f}(x_n + \dfrac{h}{2}, \mathbf{y}_n + \dfrac{h}{2}\mathbf{k}_1)\\
    \mathbf{k}_3 = \mathbf{f}(x_n + \dfrac{h}{2}, \mathbf{y}_n + \dfrac{h}{2}\mathbf{k}_2)\\
    \mathbf{k}_4 = \mathbf{f}(x_n + h, \mathbf{y}_n + h\mathbf{k}_3)
\end{gathered}
\end{equation}

% \subsection{Начальные условия}
% В симуляции используются начальные значения из таблицы \ref{tab:2}. Предполагается, что в самом начале симуляции ВП отсутствует и только начинает наполняться воздухом.

% \begin{table}[h]
%     \caption{Значения физических величин до первого шага симуляции}
%     \centering
%     \begin{tabular}{@{}llcc@{}}
%         \toprule
%         обозначение & величина & значение & СИ\\
%         \midrule
%         $A$ & коэффициент $f(q)$ & -2.756 & -\\
%         \midrule
%         $B$ & коэффициент $f(q)$ & 48.46 & -\\
%         \midrule
%         $C$ & коэффициент $f(q)$ & 2771 & -\\
%         \midrule
%         $L$ & длина ВП & 15 & м\\
%         \midrule
%         $b$ & ширина ВП & 6 & м\\
%         \midrule
%         $S$ & площадь ВП & $Lb=90$ & $\text{м}^2$\\
%         \midrule
%         $d_{max}$ & максимальный клиренс ВП & 0.7 & м\\
%         \midrule
%         $d$ & текущий клиренс ВП & 0 & м\\
%         \midrule
%         $W$ & объем ВП & $Sd=0$ & $\text{м}^3$\\
%         \midrule
%         $I_z$ & момент инерции & $2.5\cdot 10^5$ & $\text{кг}\cdot\text{м}^2$\\
%         \midrule
%         $d_{gap}$ & высота зазора & 0.01 & м\\
%         \midrule
%         $S_{gap}$ & площадь истечения & $ab\cdot d_{gap}=0.42$ & $\text{м}^2$\\
%         \midrule
%         $g$ & ускорение св. падения & 9.8 & $\text{м}/\text{с}^2$\\
%         \midrule
%         $n$ & показатель политропы & 1.4 & -\\
%         \midrule
%         $\chi$ & коэффициент истечения & 1 & -\\
%         \midrule
%         $p_a$ & атмосферное давление & $10^5$ & Па\\
%         \midrule
%         $\rho_{a}$ & плотность воздуха & $1.269$ & $\text{кг}/\text{м}^3$\\
%         \midrule
%         $l_{AC}$ & расстояние ... & 0 & м\\
%         \midrule
%         $H$ & вертикальная коорд. ц.т. судна & 0.7 & м\\
%         \midrule
%         $\varphi$ & тангаж & 0 & рад \\
%         \midrule
%         $m$ & масса судна & $1.6\cdot 10^4$ & кг\\
%         \midrule
%         $p$ & избыточное давление ВП & 0 & Па\\
%         \bottomrule
%     \end{tabular}
%     \label{tab:2}
% \end{table}

\end{document}%