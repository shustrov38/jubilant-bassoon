\documentclass{article}%

\usepackage[14pt]{extsizes}%
\usepackage[left=3.5cm,right=2.5cm,top=2.5cm,bottom=2.5cm,footskip=10mm]{geometry}%

\usepackage{cmap}%
\usepackage[T1,T2A]{fontenc}%
\usepackage[utf8]{inputenc}%
\usepackage{csquotes}%
\usepackage[russian]{babel}%

\usepackage{amsmath, amsfonts, amssymb, mathtools, systeme}%
\usepackage{graphicx, float}%
\usepackage{caption}%
\usepackage{setspace}%
\usepackage{array}%
\usepackage{booktabs}%

\usepackage{tikz}%
\usepackage{pgfplots}%

% вычисление значения в точке
\newcommand\at[2]{\left.#1\right|_{#2}}

\numberwithin{equation}{subsection}

\let\oldsection\section% Store \section
\renewcommand{\section}{% Update \section
  \renewcommand{\theequation}{\thesection.\arabic{equation}}% Update equation number
  \oldsection}% Regular \section
\let\oldsubsection\subsection% Store \subsection
\renewcommand{\subsection}{% Update \subsection
  \renewcommand{\theequation}{\thesubsection.\arabic{equation}}% Update equation number
  \oldsubsection}% Regular \subsection

\begin{document}
\section{Динамика СВП}
Исходная система уравнений, описывающая динамику СВП:
\begin{equation}
\begin{gathered}
    m\dfrac{d^2H}{dt^2}=pS-mg\\
    \dfrac{dp}{dt}=\dfrac{np_a}{W}\left(Q_{in} - Q_{out} -\dfrac{dW}{dt}\right)\\
    I_z\dfrac{d^2\varphi}{dt^2}=pS\cdot l_{AC}
\end{gathered}
\label{eq:1.1}
\end{equation}

\section{Расходно напорная характеристика}
\begin{table}[h]
    \caption{Исходные параметры центробежного нагнетателя, округленные до целой части. Точные параметры можно найти в {\it initials.xlsx}}
    \centering
    \begin{tabular}{@{}l*{10}{l}@{}}
        \toprule
        $p$, Па & 2809 & 2965 & 2902 & 2715 & 2497 & 2325 & 2060 & 1280 & 593 \\
        \midrule
        $Q, \frac{\text{м}^3}{\text{c}}$ & 0 & 6 & 11 & 17 & 22 & 25 & 28 & 34 & 38 \\
        \bottomrule
    \end{tabular}
\end{table}
\begin{figure}[H]
    \centering
    % This file was created with tikzplotlib v0.10.1.
\begin{tikzpicture}

\definecolor{darkgray176}{RGB}{176,176,176}
\definecolor{lightgray204}{RGB}{204,204,204}
\definecolor{steelblue31119180}{RGB}{31,119,180}

\begin{axis}[
legend cell align={left},
legend style={fill opacity=0.8, draw opacity=1, text opacity=1, draw=lightgray204},
tick align=outside,
tick pos=left,
x grid style={darkgray176},
xlabel={Объемный расход воздуха в ВП [$Q$, $\dfrac{\text{м}^3}{\text{с}}$]},
xlabel near ticks,
xmin=0, xmax=47.9664362355135,
xtick style={color=black},
y grid style={darkgray176},
ylabel={Избыточное давление ВП [$p$, Па]},
ylabel near ticks,
ymin=473.424819454775, ymax=3103.40709363445,
ytick style={color=black},
width=14cm,
height=7cm
]
\addplot [semithick, black, opacity=0.7]
table {%
0 2808.80274448953
5.58329944639905 2964.84734140562
11.1665988927981 2902.42950263918
16.7498983391972 2715.17598633988
22.3331977855962 2496.71355065736
25.1248475087957 2325.06449404967
27.9164972319952 2059.78867929232
33.4997966783943 1279.5656947119
37.9664362355135 592.969468281124
};
\addlegendentry{Табличные данные}
\addplot [semithick, steelblue31119180]
table {%
0 2770.84648143807
0.383499355914278 2789.02622646661
0.766998711828556 2806.39522773118
1.15049806774283 2822.95348523176
1.53399742365711 2838.70099896837
1.91749677957139 2853.637768941
2.30099613548567 2867.76379514965
2.68449549139994 2881.07907759432
3.06799484731422 2893.58361627502
3.4514942032285 2905.27741119174
3.83499355914278 2916.16046234448
4.21849291505706 2926.23276973324
4.60199227097133 2935.49433335802
4.98549162688561 2943.94515321883
5.36899098279989 2951.58522931566
5.75249033871417 2958.41456164851
6.13598969462844 2964.43315021738
6.51948905054272 2969.64099502227
6.902988406457 2974.03809606319
7.28648776237128 2977.62445334013
7.66998711828555 2980.40006685309
8.05348647419983 2982.36493660207
8.43698583011411 2983.51906258708
8.82048518602839 2983.8624448081
9.20398454194267 2983.39508326515
9.58748389785694 2982.11697795822
9.97098325377122 2980.02812888732
10.3544826096855 2977.12853605243
10.7379819655998 2973.41819945357
11.1214813215141 2968.89711909073
11.5049806774283 2963.56529496391
11.8884800333426 2957.42272707311
12.2719793892569 2950.46941541834
12.6554787451712 2942.70535999959
13.0389781010854 2934.13056081686
13.4224774569997 2924.74501787015
13.805976812914 2914.54873115947
14.1894761688283 2903.5417006848
14.5729755247426 2891.72392644616
14.9564748806568 2879.09540844354
15.3399742365711 2865.65614667694
15.7234735924854 2851.40614114637
16.1069729483997 2836.34539185182
16.4904723043139 2820.47389879328
16.8739716602282 2803.79166197078
17.2574710161425 2786.29868138429
17.6409703720568 2767.99495703382
18.0244697279711 2748.88048891938
18.4079690838853 2728.95527704096
18.7914684397996 2708.21932139856
19.1749677957139 2686.67262199219
19.5584671516282 2664.31517882183
19.9419665075424 2641.1469918875
20.3254658634567 2617.16806118919
20.708965219371 2592.3783867269
21.0924645752853 2566.77796850064
21.4759639311996 2540.36680651039
21.8594632871138 2513.14490075617
22.2429626430281 2485.11225123797
22.6264619989424 2456.26885795579
23.0099613548567 2426.61472090964
23.3934607107709 2396.14984009951
23.7769600666852 2364.87421552539
24.1604594225995 2332.78784718731
24.5439587785138 2299.89073508524
24.9274581344281 2266.18287921919
25.3109574903423 2231.66427958917
25.6944568462566 2196.33493619517
26.0779562021709 2160.19484903719
26.4614555580852 2123.24401811523
26.8449549139994 2085.4824434293
27.2284542699137 2046.91012497939
27.611953625828 2007.5270627655
27.9954529817423 1967.33325678763
28.3789523376566 1926.32870704578
28.7624516935708 1884.51341353996
29.1459510494851 1841.88737627016
29.5294504053994 1798.45059523638
29.9129497613137 1754.20307043862
30.2964491172279 1709.14480187689
30.6799484731422 1663.27578955117
31.0634478290565 1616.59603346148
31.4469471849708 1569.10553360781
31.8304465408851 1520.80428999017
32.2139458967993 1471.69230260854
32.5974452527136 1421.76957146294
32.9809446086279 1371.03609655336
33.3644439645422 1319.4918778798
33.7479433204564 1267.13691544226
34.1314426763707 1213.97120924075
34.514942032285 1159.99475927525
34.8984413881993 1105.20756554579
35.2819407441136 1049.60962805234
35.6654401000278 993.200946794911
36.0489394559421 935.981521773506
36.4324388118564 877.951352988125
36.8159381677707 819.110440438765
37.1994375236849 759.458784125427
37.5829368795992 698.996384048112
37.9664362355135 637.723240206818
};
\addlegendentry{$f(x)=Ax^2+Bx+C$}
\end{axis}

\end{tikzpicture}

    \caption{Расходно напорная характеристика $p(Q_{in})$. Аппроксимация МНК квадратичной функцией $f(q)=-2.756q^2 + 48.46q + 2771$.}
\end{figure}
Для выполнения условия устойчивости судна в предположении отсутсвия волнения, необходимо:
\begin{equation}
    \dfrac{1}{2}Q_0 - \at{\dfrac{\partial Q}{\partial p}}{0}p_0 > 0, \text{ где } (Q_0, p_0) \text{ точка на нисходящей ветви РНХ}
\end{equation} 
Преобразуем $p(Q_{in})$ для получения функции $Q_{in}(p)$, решив уравнение $f(q)=p$ и выбрав решения на нисходящей ветке. Получим:
\begin{equation}
    Q_{in}(p)=\dfrac{-B-\sqrt{B^2-4A(C-p)}}{2A}, \text{ где } A, B, C \text{ из } f(q)
\end{equation}
Объмный расход воздуха $Q_{out}$, вытекающего из зоны ВП, может быть вычислено слующим образом:
\begin{equation}
    Q_{out}=Q_{out}(p, S_{gap})=\chi\sqrt{\dfrac{2p}{\rho}}S_{gap}
\end{equation}
Объмный расход воздуха $Q_{in}$, нагнетаемого вентиляторами в зону ВП, вычисляется с помощью РНХ вентилятора:
\begin{equation}
    Q_{in}=Q_{in}(p)
\end{equation}

\section{Симуляция}
Преобразуем \ref{eq:1.1}, заменив дифференциальные уравнения второго порядка уравнениями первого порядка. Также, допишем уравнение, описывающее изменение объема ВП:
\begin{equation}
\begin{gathered}
    \dfrac{dV}{dt}=\dfrac{pS-mg}{m}\\
    \dfrac{dH}{dt} = V\\
    \dfrac{dW}{dt}=S\dfrac{dH}{dt}+S\cdot l_{AC}\dfrac{d\varphi}{dt}\\
    \dfrac{dp}{dt}=\dfrac{np_a}{W}\left(Q_{in} - Q_{out} -\dfrac{dW}{dt}\right)\\
    \dfrac{dV_{\varphi}}{dt}=\dfrac{pS\cdot l_{AC}}{I_z}\\
    \dfrac{d\varphi}{dt} = V_{\varphi}
\end{gathered}
\end{equation}

\subsection{Шаг симуляции}
В качестве метода численного интегрирования выбран метод {\it Рунге-Кутты четвертого порядка} в следующей формулировке:
\newline\par
Рассмотрим задачу Коши для системы обыкновенных дифференциальных уравнений первого порядка (далееий первого порядка (далее $\mathbf{y}, \mathbf{f}, \mathbf{k}_i \in \mathcal{R}^n, x, h \in \mathcal{R}^1$)
\begin{equation}
    \mathbf{y}'=\mathbf{f}(x, \mathbf{y}), \quad \mathbf{y}(x_0)=\mathbf{y}_0
\end{equation}
Тогда приближенное значение в последующих точдках вычисляется по итерационной формуле:
\begin{equation}
    \mathbf{y}_{n+1}=\mathbf{y}_n + \dfrac{h}{6}(\mathbf{k}_1 + 2\mathbf{k}_2 + 2\mathbf{k}_3 + \mathbf{k}_4)
\end{equation}
Вычисление нового значения происходит в четыре стадии:
\begin{equation}
\begin{gathered}
    \mathbf{k}_1 = \mathbf{f}(x_n, \mathbf{y}_n)\\
    \mathbf{k}_2 = \mathbf{f}(x_n + \dfrac{h}{2}, \mathbf{y}_n + \dfrac{h}{2}\mathbf{k}_1)\\
    \mathbf{k}_3 = \mathbf{f}(x_n + \dfrac{h}{2}, \mathbf{y}_n + \dfrac{h}{2}\mathbf{k}_2)\\
    \mathbf{k}_4 = \mathbf{f}(x_n + h, \mathbf{y}_n + h\mathbf{k}_3)
\end{gathered}
\end{equation}

\subsection{Начальные условия}
В симуляции используются начальные значения из таблицы \ref{tab:2}. Предполагается, что в самом начале симуляции ВП отсутствует и только начинает наполняться воздухом.

\begin{table}[h]
    \caption{Значения физических величин до первого шага симуляции}
    \centering
    \begin{tabular}{@{}llcc@{}}
        \toprule
        обозначение & величина & значение & СИ\\
        \midrule
        $A$ & коэффициент $f(q)$ & -2.756 & -\\
        \midrule
        $B$ & коэффициент $f(q)$ & 48.46 & -\\
        \midrule
        $C$ & коэффициент $f(q)$ & 2771 & -\\
        \midrule
        $a$ & длина ВП & 15 & м\\
        \midrule
        $b$ & ширина ВП & 6 & м\\
        \midrule
        $S$ & площадь ВП & $ab=90$ & $\text{м}^2$\\
        \midrule
        $d_{max}$ & максимальный клиренс ВП & 0.7 & м\\
        \midrule
        $d$ & текущий клиренс ВП & 0 & м\\
        \midrule
        $W$ & объем ВП & $Sd=0$ & $\text{м}^3$\\
        \midrule
        $I_z$ & момент инерции & $2.5\cdot 10^5$ & $\text{кг}\cdot\text{м}^2$\\
        \midrule
        $S_{gap}$ & площадь истечения & 0.012 & $\text{м}^2$\\
        \midrule
        $g$ & ускорение св. падения & 9.8 & $\text{м}/\text{с}^2$\\
        \midrule
        $n$ & показатель политропы & 1.4 & -\\
        \midrule
        $\chi$ & коэффициент истечения & 1 & -\\
        \midrule
        $p_a$ & атмосферное давление & $10^5$ & Па\\
        \midrule
        $\rho$ & плотность воздуха & $1.269$ & $\text{кг}/\text{м}^3$\\
        \midrule
        $l_{AC}$ & расстояние ... & 1 & м\\
        \midrule
        $H$ & вертикальная коорд. ц.т. судна & 0 & м\\
        \midrule
        $\varphi$ & тангаж & 0 & рад \\
        \midrule
        $m$ & масса судна & $1.6\cdot 10^4$ & кг\\
        \midrule
        $p$ & избыточное давление ВП & 0 & Па\\
        \bottomrule
    \end{tabular}
    \label{tab:2}
\end{table}

\end{document}%