\begin{centering}
    \subsection{Уравнение воздушной подушки}
\end{centering}

Избыточное давление в воздушной подушке $p$ представляется в виде двух слагаемых:
\begin{equation}
    p = p_{qs} + p_{damp}
\end{equation}
где $p_{qs}$ - стационарная составляющая давления, зависящая только от текущего положения судна, а $p_{damp}$ - демпфирующее давление ВП, отвеячающее ща влияние вертикальной скорости судна $\dot{H}$.

Изменение стационарной части давления можно связать с законом изменения массы воздуха в зоне ВП \cite[раздел 2.1]{shabarov2022analytical}:
\begin{equation}\label{eq:p_qs}
    \dfrac{dp_{qs}}{dt} = \dfrac{np_a}{W} \left( Q_{in} - Q_{out} - \dfrac{dW}{dt}\right)
\end{equation}
где $n$ - коэффициент политропы, а $W$ - объем воздушной подушки. Члены $Q_{in}$ и $Q_{out}$ в данном случае зависят от $p_{qs}$ и вычисляются по уравнениям (\ref{eq:q_out}) и (\ref{eq:q_in}). 

Демпфирующее давление $p_{damp}$ вычисляется следующим образом:
\begin{equation}
    p_{damp} = D(S_{gap})\rho \left( -\dfrac{Q_{in}}{S}\dot{H} + \dfrac{1}{2}\dot{H}^2 \right) S
\end{equation}
где член $D(S_{gap})$ зависит только от зазора под гибким ограждением воздушной подушки.

\begin{figure}[H]
    \centering
    % This file was created with tikzplotlib v0.10.1.
\begin{tikzpicture}

\definecolor{darkgray176}{RGB}{176,176,176}
\definecolor{lightgray204}{RGB}{204,204,204}
\definecolor{steelblue31119180}{RGB}{31,119,180}

\begin{axis}[
legend cell align={left},
legend style={fill opacity=0.8, draw opacity=1, text opacity=1, draw=lightgray204},
tick align=outside,
tick pos=left,
scaled x ticks = false,
x grid style={darkgray176},
xmin=0, xmax=0.02,
xtick style={color=black},
xtick={0,0.005,0.01,0.015,0.02},
xticklabels={0,0.005,0.01,0.015,0.02},
xlabel={Безмерная площадь зазора [$\dfrac{S_{gap}}{S}$, -]},
xlabel near ticks,
y grid style={darkgray176},
ymin=0, ymax=3000000,
ytick style={color=black},
ytick={0,1000000,2000000,3000000},
yticklabels={0,1,2,3},
ylabel={Коэф. демпфирования [$D$, -]},
width=10cm,
height=8cm
]
\addplot [semithick, steelblue31119180]
table {%
0.001 2517512.41727902
0.00119191919191919 2073453.86446567
0.00138383838383838 1708706.44098206
0.00157575757575758 1409104.77337527
0.00176767676767677 1163013.49636093
0.00195959595959596 960875.380420848
0.00215151515151515 794840.166123399
0.00234343434343434 658459.690533278
0.00253535353535354 546437.465604134
0.00272727272727273 454422.98315145
0.00291919191919192 378842.758009605
0.00311111111111111 316761.547745454
0.0033030303030303 265768.359241072
0.00349494949494949 223882.815084003
0.00368686868686869 189478.243399237
0.00387878787878788 161218.504230671
0.00407070707070707 138006.099054037
0.00426262626262626 118939.548196285
0.00445454545454545 103278.380865932
0.00464646464646465 90414.3781430899
0.00483838383838384 79847.9521186033
0.00503030303030303 71168.7438397758
0.00522222222222222 64039.6865622535
0.00541414141414141 58183.9153866526
0.00560606060606061 53374.0149010565
0.0057979797979798 49423.1872495706
0.00598989898989899 46177.9976290183
0.00618181818181818 43512.4154770408
0.00637373737373737 41322.919934487
0.00656565656565657 39524.4794972615
0.00675757575757576 38047.2497229203
0.00694949494949495 36833.860743763
0.00714141414141414 35837.189243964
0.00733333333333333 35018.5283729864
0.00752525252525253 34346.0845218228
0.00771717171717172 33793.7425826902
0.00790909090909091 33340.0517396603
0.0081010101010101 32967.3924022714
0.00829292929292929 32661.2919290601
0.00848484848484848 32409.86256637
0.00867676767676768 32203.3397741654
0.00886868686868687 32033.7030092162
0.00906060606060606 31894.3642383462
0.00925252525252525 31779.9120848088
0.00944444444444444 31685.9016714288
0.00963636363636364 31608.6819988344
0.00982828282828283 31545.2541548201
0.010020202020202 31493.1548482427
0.0102121212121212 31450.3607443593
0.0104040404040404 31415.2098863635
0.0105959595959596 31386.3371514371
0.0107878787878788 31362.621234681
0.010979797979798 31343.1411019879
0.0111717171717172 31327.1402206538
0.0113636363636364 31313.9971785863
0.0115555555555556 31303.2015510717
0.0117474747474747 31294.3340778614
0.0119393939393939 31287.0503807324
0.0121313131313131 31281.0675891739
0.0123232323232323 31276.1533547956
0.0125151515151515 31272.1168278185
0.0127070707070707 31268.8012452103
0.0128989898989899 31266.0778426183
0.0130909090909091 31263.8408536622
0.0132828282828283 31262.0034023781
0.0134747474747475 31260.4941292937
0.0136666666666667 31259.2544201035
0.0138585858585859 31258.2361293171
0.014050505050505 31257.3997104747
0.0142424242424242 31256.7126803166
0.0144343434343434 31256.1483572596
0.0146262626262626 31255.6848251877
0.0148181818181818 31255.3040823152
0.015010101010101 31254.9913420677
0.0152020202020202 31254.7344588283
0.0153939393939394 31254.5234562497
0.0155858585858586 31254.3501398111
0.0157777777777778 31254.2077785751
0.015969696969697 31254.0908437842
0.0161616161616162 31253.9947941452
0.0163535353535354 31253.9158994631
0.0165454545454545 31253.8510957748
0.0167373737373737 31253.7978663568
0.0169292929292929 31253.7541439852
0.0171212121212121 31253.7182306528
0.0173131313131313 31253.6887316254
0.0175050505050505 31253.6645012751
0.0176969696969697 31253.6445985894
0.0178888888888889 31253.6282506251
0.0180808080808081 31253.6148224909
0.0182727272727273 31253.6037926902
0.0184646464646465 31253.5947328683
0.0186565656565657 31253.587291177
0.0188484848484849 31253.5811786098
0.019040404040404 31253.5761577784
0.0192323232323232 31253.5720336931
0.0194242424242424 31253.5686461906
0.0196161616161616 31253.5658637133
0.0198080808080808 31253.5635782008
0.02 31253.5617008924
};
\end{axis}

\end{tikzpicture}

    \caption{Кривая коэффициента демпфирования в виде экспоненциальной функции $f(x) = A + B e^{Cx}$, где $A=31253.5531$, $B=6930673.3521$ и $C=1025.1779$}\label{fig:D_S_gap}
\end{figure}

Для различных площадей зазора значения коэффициента $D(S_{gap})$ определяются путем выполнения ряда независимых расчетов методом CFD. График функции $D(S_{gap})$, применяемой в данном исследовании, представлен на рис.~\ref{fig:D_S_gap}.