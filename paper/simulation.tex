\begin{centering}
    \subsection{Моделирование}
\end{centering}

Как уже было сказано ранее, описание динамики судна на воздушной подушке сводится в упрощенном случае сводится к двум уравнениям - второму закону Ньютона для сил действующих на судно и моменту вращения судна, как жесткого стержня (\ref{eq:acv_dynamics}). Динамика судна также должна учитывать текущее состояние воздушной подушки, которое в свою очередь зависит от параметров нагнетателя (Рис. \ref{fig:fan_perf_curve}) и формы волны под днищем судна. Поэтому будем использовать уравнения (\ref{eq:acv_dynamics}) и (\ref{eq:p_qs}) для дальнейшего описания судна:
\begin{equation}\label{eq:basic_dynamics}
    \begin{gathered}
        m\dfrac{d^2H}{dt^2}=\mathbf{F}\\
        \dfrac{dp}{dt} = \dfrac{np_a}{W} \left( Q_{in} - Q_{out} - \dfrac{dW}{dt}\right)\\
        I_z\dfrac{d^2\varphi}{dt^2}=\mathbf{M}
    \end{gathered}
\end{equation}
Распишем $\mathbf{F}$ и $\mathbf{M}$ из системы (\ref{eq:basic_dynamics}) c учетом знаков:
\begin{equation}
    \begin{gathered}
        \mathbf{F} = F_{AC} - F_{attr} + F_{damp} + F_{wave} = (p_{qs} + p_{damp})S - mg + F_{wave}\\
        \mathbf{M} = M_{AC} + M_{contact} = p_{qs}S \cdot l_{AC} + M_{contact}
    \end{gathered}
\end{equation}
Так как численные методы интегрирования предполагают наличиние задачи Коши для дифференциальных уравнений первого порядка, необходимо преобразовать систему второго порядка (\ref{eq:basic_dynamics}) к системе второго порядка путем разложения каждого уравнения на два уравнения меньших порядков:
\begin{equation}\label{eq:full_dynamics}
    \begin{gathered}
        \dfrac{d\dot{H}}{dt}=\dfrac{(p_{qs} + p_{damp})S - mg + F_{wave}}{m}\\
        \dfrac{dp}{dt} = \dfrac{np_a}{W} \left( Q_{in} - Q_{out} - \dfrac{dW}{dt}\right)\\
        \dfrac{d\dot{\varphi}}{dt} = \dfrac{p_{qs}S \cdot l_{AC} + M_{contact}}{I_z}
    \end{gathered}
    \qquad
    \begin{gathered}
        \dfrac{dH}{dt} = \dot{H}\\
        \dfrac{d\varphi}{dt}=\dot{\varphi}
    \end{gathered}
\end{equation}
Таким образом, систему дифференциальных уравнений (\ref{eq:full_dynamics}) можно интегрировать с помощью распространенных численных методов.

\begin{centering}
    \subsubsection{Метод Эйлера}
\end{centering}
В качестве простейшнего как с точки зрения идеи, так и с точки зрения практической реализации в программном коде, будем использовать явный метод Эйлера.

Пусть дана задача Коши для уравнения первого порядка:
\begin{equation}
    \begin{gathered}
        \dfrac{dy}{dx} = f(x, y)\\
        y|_{x=x_0}=y_0
    \end{gathered}
\end{equation}
Функция $f:\mathcal{R}^2 \to \mathcal{R}$. Решение такой задачи ищется на полуинтервале $(x_0, b]$. На промежутке вводятся точки $x_0 < x_1 < \dots < x_N \leq b$ чаще всего расстояние между которыми одинаково. Тогда приближенное значение $y_i$ в узлах $x_i$, находится следующим образом:
\begin{equation}
    y_i = y_{i - 1} + (x_i - x_{i - 1}) \cdot f(x_{i - 1}, y_{i - 1}),\,\forall i = 1,\dots,N
\end{equation}
Очевидно, что при такой интерпритации метода интегрирования, чем больше параметр $N$, тем более точным будет являться полученный результат. Также стоит отметить, что явный метод Эйлера сильно зависим от ошибок округления и может быть численно неустойчивым.
Метод Эйлера имеет множество более сильных аналогов и модификаций: {\it обратный}, {\it неявный}, {\it полуявный} методы Эйлера и семейство методов {\it Рунге-Кутты} разных порядков точности. Но в настоящей работе будет использоваться обыкновенный явный метод Эйлера в целях уменьшения количества операций с плавающей точкой на единицу времени и достижения большей производительности вычислений.

\begin{centering}
    \subsubsection{Общие детали моделирования}
\end{centering}

В данном разделе будут приведены все примененные допущения реализации, которые помогли получить необходимый результат (Значения используемых параметров приведены в таблице \ref{tab:total_params}). 

Для простоты проведения испытаний предполагается, что судно идет под действием буксира с постоянной скоростью. Также, в стартовой точке судно находится на подсдутой воздушной подушке, и выходит в режимное состояние лишь спустя некоторое время после начала моделирования.

Нужно отметить, что хоть в математической модели динамики СВП и присутствует параметр сдвига центра тяжести, в данной работе он не применяется, так как в реальных судах обычно присутствует большой вентилятор, установленный в корме судна, который создает обратный момент, который будет закручивать судно вперед. Для решения этой проблемы изменяют положение центря тяжести, в целях уменьшения плеча закручивающей силы.

В настоящей работе также вводятся {\it сдерживающие} ограничения, которые на реальном судне обеспечиваются геометрией огражений и физическими свойствами центробежного нагнетателя - ограничение на максимальное и минимальное избыточное давление воздушной подушки, а также ограничение на максимальный и минимальный тангаж судна.

Численное интегрирование производилось сначала с шагом $dt = 10^{-4}$, но как показали результаты экспериментов, это недостаточно, поскольку в силу программной реализации, одни физические процессы начинают протекать моментально, в то время как другие только начинают накапливать энергию и ускорение за счет дифференциальных уравнений (пример моментального процесса - изменение нагнетаемого вентилятором воздуха, изменение вертикальной координаты - интегрируемый процесс). Для решения этой проблемы был выбран более мелкий шаг $dt = 10^{-7}$. С таким шагом моментальные изменения параметров нагнеталя перестают быть моментальными в рамках моделирования и становятся более отзывчивыми, так как за минимальным изменением интегрируемой величины следует такое же минимальное изменение нагоняемого в ВП объема воздуха.

Замер состояния системы происходил каждую секунду моделирования (один раз за $10^7$ итераций). В момент проведения замера сохранялись важные для проведения исследования физические величины, такие как: вертикальная скорость, тангаж, изменение вертикальной координаты, состояние ВП и прочие. Такой шаг записи был выбран по нескольким причинам. Во-первых, это позволяет обойтись без перегрузки дискового пространства, ведь рассчеты требуют минимум 100 секунд моделирования, что с частым сохранением может занимать значительное количество дискового пространства. Также при маленьком интервале записи визуальная репрезентация результатов моделирования становится нечитаемой, так как достаточно большое колитчество точек начинает перекрывать друг друга.

Также для определения степени влияния коэффициента демпфирования на поведение и динамику судна на воздушной подушке во время моделирования введем целевую функцию - вертикальную перегрузку, опреляемую как отношение вертикального ускорения СВП к ускорению свободного падения. Целью экспериментов станет уменьшение ветикальной перегрузки и уменьшение колебаний тангажа при различных длинах волны.

\begin{table}[h]
    \caption{Значения физических величин и параметров судна до первого шага моделирования}\label{tab:total_params}
    \centering
    \begin{tabular}{@{}llcc@{}}
        \toprule
        обозначение & величина & значение & СИ\\
        \midrule
        $L$ & длина ВП & 15 & м\\
        \midrule
        $b$ & ширина ВП & 6 & м\\
        \midrule
        $S$ & площадь ВП & $Lb=90$ & $\text{м}^2$\\
        \midrule
        $d_{max}$ & максимальный клиренс ВП & 0.7 & м\\
        \midrule
        $d$ & текущий клиренс ВП & 0 & м\\
        \midrule
        $W$ & объем ВП & $Sd=0$ & $\text{м}^3$\\
        \midrule
        $I_z$ & момент инерции & $2.5\cdot 10^5$ & $\text{кг}\cdot\text{м}^2$\\
        \midrule
        $d_{gap}$ & высота зазора & 0.01 & м\\
        \midrule
        $S_{gap}$ & площадь истечения & $ab\cdot d_{gap}=0.42$ & $\text{м}^2$\\
        \midrule
        $g$ & ускорение св. падения & 9.8 & $\text{м}/\text{с}^2$\\
        \midrule
        $n$ & показатель политропы & 1.4 & -\\
        \midrule
        $\chi$ & коэффициент истечения & 1 & -\\
        \midrule
        $p_a$ & атмосферное давление & $10^5$ & Па\\
        \midrule
        $\rho_{a}$ & плотность воздуха & $1.269$ & $\text{кг}/\text{м}^3$\\
        \midrule
        $l_{AC}$ & сдвиг ц.т. судна & 0 & м\\
        \midrule
        $H$ & вертикальная коорд. ц.т. судна & 0.7 & м\\
        \midrule
        $\varphi$ & тангаж & 0 & рад \\
        \midrule
        $m$ & масса судна & $1.6\cdot 10^4$ & кг\\
        \midrule
        $p$ & избыточное давление ВП & 0 & Па\\
        \bottomrule
    \end{tabular}
    \label{tab:2}
\end{table}