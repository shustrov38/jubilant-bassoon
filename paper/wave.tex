\begin{centering}
    \subsection{Волновой процесс}
\end{centering}

В разделе \ref{sec:fan_performance} было описано, что нет необходимости рассматриывать динамику и свойства устойчивости СВП на стоячей воде, поэтому предлагается ввести в моделирование волновой процесс $\Psi(x, t)$. В настоящей работе будет рассматриваться {\it регулярное волнение} - волнение, при котором форма профиля всех волн одинакова и схожа с синусоидой.

\begin{centering}
    \subsubsection{Регулярная волна}
\end{centering}

Волновой процесс $\Psi(x, t)$ можно характеризовать синусоидой (или косинусоидой) с длиной волны $\lambda$:
\begin{equation}
    \Psi(x, t) = A \cos(\dfrac{2\pi x}{\lambda}+2\pi \nu t + \varphi_0) = A \cos(kx + \omega t + \varphi_0)
\end{equation}
где $k=\dfrac{2\pi}{\lambda}$ - волновой вектор, а $\omega=kc$ - циклическая частота. Волна бежит влево. Скорость волны $c$ вычисляется следующим способом:
\begin{equation}
    c = \sqrt{\frac{\lambda}{2\pi}g}=\sqrt{\dfrac{g}{k}}
\end{equation}
Таким образом, зная время прошедшее с начала моделирования можно вычислять высоту волны в различных позициях и разные моменты времени:
\begin{equation*}
    \Psi(x_0, dt) \quad \Psi(x_0, 2 \cdot dt) \quad \dots \quad \Psi(x_0, n \cdot dt) 
\end{equation*}
Далее в работе для удобства обозначения вместо $\Psi(x, t)$ будет использоваться обозначение $y_{wave}$ (почти всегда рассматривается часть волны, находящаяся в конкретный момент времени под судном).