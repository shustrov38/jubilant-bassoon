\begin{centering}
    \subsection{Удары волн о корпус СВП}
\end{centering}

Для учета удар волн о корпус будем использовать {\it метод плоских поперечных сечений}. Для этого необходимо разбить корпус на $N$ сечений (чем больше $N$, тем точнее будут вычисления). Тогда расстояние между сечениями:
\begin{equation}
    \Delta L = \dfrac{L}{N}
\end{equation}
Для каждого сечения необходимо вычислить положение свободной поверхности $y$. На данном этапе $y=0$. Если в каком то из сечений вода касается корпуса судна ($y_{ship}<y_{wave}$), то пятно контакта имеет площадь:
\begin{equation}
    S^i_{wash} = S^i_{wash}(\Delta L) = \Delta L \cdot b
\end{equation}
где $i$ - номер сечения, a $b$ - ширина судна. Тогда сила, действующая на пятно контакта определяется как:
\begin{equation}\label{eq:f_contact^i}
    F^i_{contact} = \theta \cdot \rho_w \dfrac{V_i^2}{2}S^i_{wash}
\end{equation}
где $V_i$ - скорость $i$-го сечения, т.е. $V_i=|\mathbf{V}+\mathbf{w}\times\mathbf{r}_i|$, где $\mathbf{w}$ - вектор угловой скорости судна, $\mathbf{r}_i$ - вектор, соединяющий центр тяжести судна с $i$-м сечением. Тогда $V_i=\sqrt{V_x^2+(V_y + w_z \cdot x_i)^2}$, где $x_i$ - координата $i$-того сеченияв системе координат, связанной с судном (начало этой системы координат в центе тяжести судна, ость $x$ напавлена к носу, $z$ к правому борту, $y$ наверх). $\theta$- гиперпараметр, отвечающий за пропорциональное увеличение силы удара волны.

Таким образом получим выражение для полной силы, которую вызывает удар волны:
\begin{equation}
    F_{wave} = \sum_{i=1}^{N} F^i_{contact}
\end{equation}
Также необходимо вычислить момент вращения судна под влиянием удара волны о корпус:
\begin{equation}
    M_{contact} = \sum_{i=1}^{N} F^i_{contact} \cdot x_i
\end{equation}
где $x_i$ из уравнения (\ref{eq:f_contact^i}).

\begin{centering}
    \subsubsection{Влияние на объем ВП и площадь зазора}
\end{centering}
Объем воздушной подушки $W$ и площадь зазора $S_{gap}$ определяются исходя из {\it гипотезы плоских поперечных сечений}, как и замытая часть корпуса. Принимая уровень волны в каждом сечении за $y_{wave}$, а высоту жесткого корпуса судна за $y_{ship}$, получим расстоение от корпуса судна до волны:
\begin{equation}
    d_i=y_{ship} - y_{wave}
\end{equation}
Тогда объем в сечении равен:
\begin{equation}
    W^i =
    \begin{cases}
        d_i \cdot \Delta L \cdot b & d_i > 0 \\
        0 & d_i \leq 0
    \end{cases}
\end{equation}
Суммарный объем ВП определяется как:
\begin{equation}\label{eq:w_all}
    W = \sum_{i=1}^{N} W^i
\end{equation}
Площадь зазора $S_{gap}$ опреляется аналогичным образом, но нужно рассмотреть один крайний случай - сегменты с индексами $i=1$ и $i=N$, так как они прибавляют к результату зазор носовой и кормовой ширины судна:
\begin{equation}
    S^i_{gap} =
    \begin{cases}
        \begin{cases}
            (2 \cdot \Delta L + b) \cdot (d_i - d_{max}) & i=1,N \\
            (2 \cdot \Delta L) \cdot (d_i - d_{max}) & i=2,\dots,N-1 
        \end{cases} & d_i > d_{max} \\
        0 & d_i \leq d_{max}
    \end{cases}
\end{equation}
Суммарная высота зазора:
\begin{equation}\label{eq:s_gap_all}
    S_{gap} = \sum_{i=1}^{N} S^i_{gap}
\end{equation}
Формула для вычисления объема воздуха в сечении воздушной подушки (\ref{eq:w_all}) будет использоваться для определения необходимого объема воздуха, который необходимо нагнать под днище судна для поддержания его над поверхностью. Этот объем воздуха напрямую влияет на подъемную силу судна и его способность двигаться над водой или другой поверхностью. Путем вычисления объема воздуха в сечении воздушной подушки можно оптимизировать работу системы воздушной подушки и обеспечить эффективное движение судна. Также формула (\ref{eq:s_gap_all}) нужна для корректной регулировки избыточного давления воздушной подушки под действием волнового процесса.